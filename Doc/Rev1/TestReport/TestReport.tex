\documentclass[12pt, titlepage]{article}

\usepackage{booktabs}
\usepackage{tabularx}
\usepackage{hyperref}
\hypersetup{
    colorlinks,
    citecolor=black,
    filecolor=black,
    linkcolor=red,
    urlcolor=blue
}
\usepackage[round]{natbib}
\usepackage{indentfirst}
\usepackage{float}
\usepackage{adjustbox}

\title{SE 3XA3: Test Report\\Rummy For Dummies}

\author{Lab 2 Group 7, Rummy For Dummies
		\\ Joy Xiao, xiaoz18
		\\ Benson Hall, hallb8
		\\ Smita Singh, sings59
}

\date{April 09, 2021}

\begin{document}

\maketitle

\pagenumbering{roman}
\tableofcontents
\listoftables
\listoffigures

\begin{table}[bp]
\caption{\bf Revision History}
\begin{tabularx}{\textwidth}{p{3cm}p{2cm}X}
    \toprule {\bf Date} & {\bf Version} & {\bf Notes}\\
    \midrule
    April 1, 2021 & 1.0 & Started on the test report\\
    April 9, 2021 & 1.1 & Finished the test report\\
    \bottomrule
\end{tabularx}
\end{table}

\newpage 

\pagenumbering{arabic}

This document contains the results of our testing based off of the test plan.

\section{Functional Requirements Evaluation}
% Card Tests
\subsection{Card Tests}
\begin{enumerate}

    
    \item FR-C-1: Test accessor for card suit
    					
    Initial State: Card = new Card(Suit.D, 1)
    					
    Input: Card
    					
    Expected Output: Suit.D
    
    Results: Passed
    
    
    \item FR-C-2: Test constructor with invalid rank
    					
    Initial State: N/a
    					
    Input:  Card = new Card(Suit.D, -1)
    					
    Expected Output: IllegalArgumentException
    
    Results: Passed
    
   \item FR-C-3: Test accessor for point value of card (I)
   
    Initial State: Card  = new Card(Suit.D, 11)
    					
    Input: Card
    					
    Expected Output: 10
    
    Results: Passed					

    \item FR-C-4: Test accessor for point value of card (II)
    					
    Initial State: Card  = new Card(Suit.C, 2)
    					
    Input: Card
    					
    Expected Output: 2
    					
    Results: Passed
    
    \item FR-C-5: Test String representation of card (I)
    					
    Initial State: Card  = new Card(Suit.H, 11)
    					
    Input: Card
    					
    Expected Output: "Jh"
    					
    Results: Passed
    
    \item FR-C-6: Test String representation of card (II)
    					
    Initial State: Card = new Card(Suit.C, 2)
    					
    Input: Card
    					
    Expected Output: "2c"
    
    Results: Passed
    
    
    \item FR-C-7:  Test Symbol representation of card
    					
    Initial State: Card = new Card(Suit.S, 11)
    					
    Input: Card
    					
    Expected Output: "J spade symbol"
    
    Results: Passed
    
    
    \item FR-C-8: Test getRank method
    					
    Initial State: Card = new Card(Suit.D, 1)
    					
    Input: Card
    					
    Expected Output: 1
    
    Results: Passed

    
\end{enumerate}

\subsection{Discard Pile Tests}
% Discard Pile Test %
\begin{enumerate}
    \item FR-DP-1: Test display of discard pile
    
    Initial State: DiscardPile = [5h, 9d]
    
    Input: DiscardPile.displayTopCard
    
    Expected Output: 9h displayed
    					
    Results: Pass
    
    \item FR-DP-2: Test drawing card from discard pile
    
    Initial State: DiscardPile = [5h, 9d]

    Input: DiscardPile.pop
        
    Expected Output: DiscardPile = [5h]
    					
    Results: Pass
\end{enumerate}

\subsection{Stock Pile Tests}
% Stock Pile Tests %
\begin{enumerate}
    \item FR-SP-1: Test drawing from stock pile with 1 or more cards
    
    Initial State: StockPile = [As, 5d]
    
    Input: Stockpile.pop
    
    Expected Output: Stock pile = [As]
    
    Results: Pass

\end{enumerate}

\subsection{Computer Tests}
% Computer Tests %
\begin{enumerate}
    \item FR-CP-1: Test computer make move method (I)
    					
    Initial state: DiscardPile = [4h, 9c, 5d], Computer Hand = [4d, 6d, Ks, Js, Ad, 2d], StockPile = [Qs, 3d]
    		
    Input: Computer.makeMove
    
    Expected Output: DiscardPile = [4h, 9c, Ks], Computer Hand = [4d, 6d, Js, Ad, 2d, 5d], StockPile = [Qs, 3d]. Returns False.
    					
    Results: Pass
    
    \item FR-CP-2: Test computer make move method (II)
    					
    Initial state: DiscardPile = [4h, Jc, 10d, Ks], Computer Hand = [4d, 6d, Js, Ad, 2d, 5d], StockPile = [Qs, 3d]
    		
    Input: Computer.makeMove
    
    Expected Output: DiscardPile = [4h, Jc, 10d, Ks, Js], Computer Hand = [4d, 6d, Ad, 2d, 5d, 3d], StockPile = [Qs]. Returns False.
    					
    Results: Pass
    
    \item FR-CP-3: Test computer make move method (III)
    
    Initial state: DiscardPile = [4h, Jc, 10d, As], Computer Hand = [4d, 6d, Js, Ad, 2d, 5d], StockPile = [Qs, 3d]
    		
    Input: Computer.makeMove
    
    Expected Output: DiscardPile = [4h, Jc, 10d, Js], Computer Hand = [4d, 6d, Ad, 2d, 5d, As], StockPile = [Qs, 3d]. Returns False.
    					
    Results: Pass
    
    \item FR-CP-4: Test that computer never discards a meld card
    
    Initial state: DiscardPile = [4h, Jc, 10d, As], Computer Hand = [4d, 6d, Ad, 2d, Kc, Kd, Ks], StockPile = [Qs, 3d]

    Input: Computer.makeMove
    
    Expected Output: DiscardPile = [4h, Jc, 10d, 6d], Computer Hand = [4d, Ad, 2d, Kc, Kd, Ks, As], StockPile = [Qs, 3d]. Returns False.
    
    Results: Pass
    					
    \item FR-CP-5: Test that computer never discards a meld card
    
    Initial state: DiscardPile = [4h, Jd, 10d, As], Computer Hand = [4d, 6d, Ad, 2d, Jc, Qc, Kc], StockPile = [Qs, 3d]

    Input: Computer.makeMove
    
    Expected Output: DiscardPile = [4h, Jd, 10d, 6d], Computer Hand = [4d, Ad, 2d, Jc, Qc, Kc, As], StockPile = [Qs, 3d]. Returns False.
    
    Results: Pass

    \item FR-CP-6: Test that computer knocks whenever it's hand has deadwood score of 10 or less
    
    Initial state: Computer Hand = [4d, Ad, 2d, Jc, Qc, Kc, As]
    
    Input: computer.makeMove
    
    Expected Output: Returns True.
    
    Results: Pass
    
    \item FR-CP-7: Test computer move when discard pile has 1 card
    
    Initial state: DiscardPile = [4h], StockPile = [Qs, 3d], Computer Hand = [4d, 6d, Js, Ad, 2d, 5d]
    
    Input: computer.makeMove
    
    Expected Output: DiscardPile = [Js], StockPile = [Qs, 3d], Computer Hand = [4d, 6d, Ad, 2d, 5d, 4h]. Returns False.
    
    Results: Pass
    
    \item FR-CP-8: Test computer does not draw from empty stock pile
    
    Initial state: DiscardPile = [4h], StockPile = [], Computer Hand = [4d, 6d, Js, Ad, 2d, 5d]
    
    Input: computer.makeMove
    
    Expected Output: DiscardPile = [Js], StockPile = [], Computer Hand = [4d, 6d, Ad, 2d, 5d, 4h]. Returns False.
    
    Results: Pass

\end{enumerate}

\subsection{Hand Tests}
% Hand Test %
\begin{enumerate}
    \item FR-H-1: Test remove card from hand (I)
    
    Initial state: hand = [As, 2s, Qs, 3d, 5d]
    
    Input: "As"
    
    Expected Output: hand = [2s, Qs, 3d, 5d]
    
    Results: Pass
    
    \item FR-H-2: Test remove card from hand (II)
    
    Initial state: hand = [As, 2s, Qs, 3d, 5d]
    
    Input: "Kc"
    
    Expected Output: hand = [As, 2s, Qs, 3d, 5d]
    
    Results: Pass
    
    \item FR-H-3: Test remove card from hand (III)
    
    Initial state: hand = [As, 2s, Qs, 3d, 5d]
    
    Input: "AS"
    
    Expected Output: hand = [2s, Qs, 3d, 5d]
    
    Results: Pass
    
    \item FR-H-4: Test remove card from hand (IV)
    
    Initial state: hand = [As, 2s, Qs, 3d, 5d]
    
    Input: "KS"
    
    Expected Output: hand = [As, 2s, Qs, 3d, 5d]
    
    Results: Pass
    
    \item FR-H-5: Test displaying of hand
    
    Initial state: Player object initialized with a non-empty hand
    
    Input: Hand
    
    Expected Output: All cards in player's hand is displayed correctly
    
    Results: Pass
    
    \item FR-H-6: Test contains method (I)
    
    Initial state: hand = [As, 2s, Qs, 3d, 5d]
    
    Input: input = "As"
    
    Expected Output: Return True
    
    Results: Pass
    
    \item FR-H-7: Test contains method (II)
    
    Initial state: hand = [As, 2s, Qs, 3d, 5d]
    
    Input: input = "Ks"
    
    Expected Output: Return False
    
    Result: Pass
    
    \item FR-H-8: Test contains method (III)
    
    Initial state: hand = [As, 2s, Qs, 3d, 5d]
    
    Input: input = "AS"
    
    Expected Output: Return True
    
    Results: Pass
    
    \item FR-H-9: Test contains method (IV)
    
    Initial state: hand = [As, 2s, Qs, 3d, 5d]
    
    Input: input = "KS"
    
    Expected Output: Return False
    
    Results: Pass
\end{enumerate}

\subsection{Meld Tests}
% Meld Test %
\begin{enumerate}
    \item FR-M-1: Test sequence melds with more than 3 cards
    
    Initial state: hand = Player hand has 4 cards in consecutive rank with same suit eg. 3S, 4S, 5S, 6S
    
    Input: input = Hand : [3s, 7d, Qh, 4s, Kc, Kd, Ac, Ad, 5s, 6s]
    
    Expected Output: [[3S, 4S, 5S, 6S]]
    
    Results: Passed
    
    \item FR-M-2: Test melds with 3 cards of same rank
    
    Initial state: Hand has  3 cards of same rank
    
    Input: input = Hand: [3s, 7d, Qh, 4s, Kc, Kd, Ac, Ad, Ah, 6s]
    
    Expected Output: [[Ah, Ac, Ad]]
    
    Results: Passed
    
    \item FR-M-3: Test melds with 4 cards of same rank
    
    Initial state: Player's hand has 4 cards of same rank
    
    Input: Hands: [Kd,5d,4d,5c,Jc,Qd,5h,5s,3s,Js]
    
    Expected Output:[[5h,5s,5c,5d]]
    
    Results: Passed
    
    \item FR-M-4: Test meld with 2 sequence melds and one group meld
    
    
    Initial state: Player's hand has 2 sequence melds and one group meld
    
    Input: Hand with Cards: [3s,4s,5s, Kc,Jc,Qc,Ac,Ad,Ah,As]
    
    Expected Output: [[3s,4s,5s],[Jc,Kc,Qc],[Ah,As,Ac,Ad]]
    
    Results: Passed
    
   \item FR-M-5: Test meld with 1 sequence and 1 group meld
    
    Initial state: Player's hand has 1 sequence melds and 1 group meld
    
    Input:  Hand : [10c,Js, Ks, 10d, Qs, 10h, Ac, 4c, 6h, 9h ]
    
    Expected Output:[[Js,Qs,Ks],[10h,10c,10d]]
    
    Results: Passed
    
    \item FR-M-6: Test Melds with overlapping cards
    
    Initial state: Player's hand has a meld that can belong to a sequence and group meld
    
    Input: [Ac, Ad, Ah, As,2s,3s,Ks,Jh, Jd, 7s]
    
    Expected Output: [[As,2s,3s],[Ah,Ac,Ad]]
    
    Results: Passed

\end{enumerate}

\subsection{Player Tests}
% Player Tests
\begin{enumerate}
    \item FR-P-1: Test accessor for player's name 
    
    Initial State: Player p = new Player("P1");
    
    Input: p
    
    Expected Output: "P1"
    
    Results: Passed
    
    \item FR-P-2: Test adding a card to player's hand
    
    Initial State: p.hand = [Ah, 2s, Qs, 3d, 5d];
    
    Input: p, new Card(Suit.S, 13)
    
    Expected Output: p.hand = [Ah, 2s, Qs, 3d, 5d, Ks];
    
    Results: Passed
    
    \item FR-P-3: Test accessor for player's hand
    
    Initial State: p.hand = [Ah, 2s, Qs, 3d, 5d];
    
    Input: p
    
    Expected Output: [Ah, 2s, Qs, 3d, 5d]
    
    Results: Passed
    
    \item FR-P-4: Test accessor for player's total score
    
    Initial State: p.totalScore = 12;
    
    Input: p
    
    Expected Output: 12
    
    Results: Passed
    
    \item FR-P-5: Test accessor for player's deadwood score
    
    Initial State: p.hand = [Ah, Ac, Ad, As, 2s, 2d, 2c, 3d, Qs, 9h];
    
    Input: p
    
    Expected Output: 22
    
    Results: Passed
    
    \item FR-P-6: Test accessor for player's melds
    
    Initial State: p.hand = [Ah, Ac, Ad, As, 2s, 2d, 2c, 3d, Qs, 9h];
    
    Input: p
    
    Expected Output: [[As, Ah, Ac, Ad], [2s, 2d, 2c]]
    
    Results: Passed
    
    \item FR-P-7: Test discarding a card that exists in the player's hand
    
    Initial State: p.hand = [Ah, 2s, Qs, 3d, 5d];
    
    Input: p, "5d"
    
    Expected Output: p.hand = [Ah, 2s, Qs, 3d], new Card(Suit.D, 5) returned
    
    Results: Passed
    
    \item FR-P-8: Test discarding a card that does not exist in the player's hand
    
    Initial State: p.hand = [Ah, 2s, Qs, 3d, 5d]
    
    Input: p, "jh"
    
    Expected Output: IllegalArgumentException()
    
    Results: Passed
    
    \item FR-P-9: Test adding a score to the total score
    
    Initial State: p.totalScore = 12
    
    Input: p, 15
    
    Expected Output: p.totalScore = 27
    
    Results: Passed
    
    \item FR-P-10: Test getting deadwood cards from the player's hand
    
    Initial State: p.hand = [Ah, Ac, Ad, 2s, 9h, 10h, Jd]
    
    Input: p
    
    Expected Output: [2s, 9h, 10h, Jd]
    
    Results: Passed
    
    \item FR-P-11: Test recalculation of deadwood score
    
    Initial State: p.hand = [Ah, Ac, Ad, 2s, 9h, 10h, Jd], p.deadwoodScore = 0
    
    Input: p
    
    Expected Output: p.deadwoodScore = 31
    
    Results: Passed
    
    \item FR-P-12: Test accessor for player's melds after checking for melds
    
    Initial State: p.hand = [As, 2s, 3s, 4d, 5d, 6d, 7d, 10h], p.melds = [[As, 2s, 3s]]
    
    Input: p
    
    Expected Output: p.melds = [[As, 2s, 3s], [4d, 5d, 6d, 7d]]
    
    Results: Passed
    
    \item FR-P-13: Test resetting hand
    
    Initial State: p.hand = [Ah, Ac, Ad, 2s, 9h, 10h, Jd]
    
    Input: p
    
    Expected Output: p.hand = []
    
    Results: Passed
    
    \item FR-P-14: Test resetting deadwood score
    
    Initial State: p.deadwoodScore = 31
    
    Input: p
    
    Expected Output: p.deadwoodScore = 0
    
    Results: Passed
    
    \item FR-P-15: Test resetting list of melds
    
    Initial State: p.melds = [[Ah, Ad, Ac]]
    
    Input: p
    
    Expected Output: p.melds = []
    
    Results: Passed
    
    \item FR-P-16: Test resetting total score
    
    Initial State: p.totalScore = 12
    
    Input: p
    
    Expected Output: p.totalScore = 0
    
    Results: Passed
\end{enumerate}

\subsection{UserInputOps Tests}
% UserInputOps Tests
\begin{enumerate}
    \item FR-UIO-1: Test that user inputting valid inputs for deciding on knocking results in successful termination
    
    Initial State: N/A
    
    Input: User input: 'Y'
    
    Expected Output: 'y'
    
    Results: Passed
    
    \item FR-UIO-2: Test that user inputting invalid inputs for deciding on knocking is handled properly
    
    Initial State: N/A
    
    Input: User inputs: ’h’, ’e’, ’l’, ’l’, ’o’, ”othello”, ”No”
    
    Expected Output: 'n'
    
    Results: Passed
    
    \item FR-UIO-3: Test that user inputting valid inputs for deciding on playing a new game of Gin-Rummy results in successful termination
    
    Initial State: N/A
    
    Input: User input: 'Y'
    
    Output: 'y'
    
    Results: Passed
    
    \item FR-UIO-4: Test that the user inputting invalid inputs for deciding on playing a new game of Gin-Rummy is handled properly
    
    Initial State: N/A
    
    Input: User inputs:  ”To”, ”be”, ”or”, ”not”, ”to”, ”Be”, ”yEs”
    
    Expected Output: ’n’ - note that ’not’ begins with n
    
    Results: Passed
    
    \item FR-UIO-5: Test/Simulate user making a valid decision in a single round
    
    Initial State:  N/A
    
    Input: User input: 3
    
    Expected Output: 3
    
    Results: Passed

    \item FR-UIO-6: Test that when the user making a invalid decision in a single round, it is handled appropriately
    
    Initial State: N/A
    
    Input: User inputs: -1, 45, "hello", 'y', "tomorrow and tomorrow", 2
    
    Expected Output: 2
    
    Results: Passed
    
    \item FR-UIO-7: Test username is received properly
    
    Initisal State: N/A
    
    Input: User input: "AC"
    
    Expected Output: "AC"
    
    Results: Passed
\end{enumerate}

\subsection{GameOps Tests}
% GameOps Tests
\begin{enumerate}
    \item FR-GO-1: Test that score is calculated correctly and given to the right player (I)
    
    Initial State: (Player) p1.deadwoodScore = 10, p1.totalScore = 0, (Player) p2.deadwoodScore = 15, p2.totalScore = 0

    Input: p1, p2

    Expected Output: p1.totalScore = 5, p2.totalScore = 0

    Results: Passed
    
    \item FR-GO-2: Test that score is calculated correctly and given to the right player (II)
    
    Initial State: p1.deadwoodScore = 20, p1.totalScore = 0, p2.deadwoodScore= 9, p2.totalScore = 0
    
    Input: p1, p2
    
    Expected Output: p1.totalScore = 0, p2.totalScore = 11

    Results: Passed
    
    \item FR-GO-3: Test that going Gin works correctly
    
    Initial State: p1.deadwoodScore = 0, p1.totalScore = 0, p2.deadwoodScore= 15, p2.totalScore = 0
    
    Input: p1, p2
    
    Expected Output: p1.totalScore = 35, p2.totalScore = 0
    
    Results: Passed

    \item FR-GO-4: Test that stock pile is created correctly
    
    Initial State: N/A
    
    Input: N/A
    
    Expected Output: Stock Pile with 52 unique cards (A to K, all suits)

    Results: Passed
    
    \item FR-GO-5: Test that discard pile is created correctly
    
    Initial State: N/A
    
    Input: N/A
    
    Expected Output: Empty stack/discard pile
    
    Results: Passed
    
    \item FR-GO-6: Test that opening distribution of cards is done correctly
    
    Initial State: StockPile sp = createStockPile();
    
    Input: N/A

    Expected Output: Player and cpu hands of size 10, discard pile of size 1, stockpile contains the rest of the cards

    Results: Passed
    
    \item FR-GO-7: Test that opening distribution of cards will only be  done with a 52-card stock pile 
    
    Initial State: Stock pile of size 28
    
    Input: N/A
    
    Expected Output: Rejected status
    
    Results: Passed. IllegalArgumentException raised.
    
    \item FR-GO-8: Test interfacing playAgain method
    
    Initial State: N/A
    
    Input: User inputs: ’A’, ’B’, ’n’
    
    Expected Output: 'n'
    
    Results: Passed
    
    \item FR-GO-9: Test process decision method
    
    Initial State: New game has started
    
    Input: User’s decision based on possible moves
    
    Expected Output: Depends on move made - 1 will draw a card from the stockpile, 2 will draw a card from the discard pile, 3 will show melds, 4 will show deadwood score and prompt user if they wish to knock. If user knocks, return true, else return false
    
    Results: All combinations of inputs have passed
    
    \item FR-GO-10: Test reset for a new deal
    
    Initial State: p.hand = [Ah, Ac, Ad, 6s], p.deadwoodScore = 6, p.melds= [Ah, Ac, Ad]
    
    Input: N/A
    
    Expected Output: p.hand = [], p.melds = [], p.deadwoodScore = 0

    Results: Passed
    
    \item FR-GO-11: Test interfacing username method
    
    Initial State: N/A
    
    Input: ”AC”
    
    Expected Output: ”AC”

    Results: Passed
    
    \item FR-GO-12: Test decision making when stock pile is empty
    
    Initial State: (StockPile) sp = new StockPile()
    
    Input: User inputs 1 when prompted
    
    Output: User should not be allowed to draw from the stock pile, and is prompted to make a new decision
    
    Results: Passed. User is forced to draw from the discard pile to carry on.
\end{enumerate}

\section{Nonfunctional Requirements Evaluation}

\subsection{Look and Feel}
\paragraph{NFR-LF-1: Test game appearance\\}

Initial state: Cards dealt, player's hand and discard pile displayed

Input: N/A

Output: Cards displayed have a simplistic look

Results: Passed. The cards visually look like standard, real-life playing cards, hence it is intuitive for the user.

\subsection{Usability}
\paragraph{NFR-UH-1: Test game usability\\}

Initial state: Cards dealt, player's hand and discard pile displayed

Input: N/A

Expected Output: Easy to set up and play the game

Results: Passed. Batch file was created and the user is no longer required to compile and run files on the command line to play game.

\subsection{Performance}
\paragraph{NFR-P-1: Test performance speed\\}

Initial state: Start a new game 

Input: User interactions with system

Expected Output: System should respond within 0.5 seconds of user input, according to the non-functional requirements

Results: Passed. The game responds immediately to player input. No noticeable delay.

\paragraph{NFR-P-2: Test game rules requirements\\}

Initial state: N/A

Input: N/A

Expected Output: Legal moves (according to game rules) can be done without errors. User is prompted for another input if they are making an illegal move.

Results: Passed. No illegal moves can successfully be made.

\subsection{Operational and Environmental}

\paragraph{NFR-OE-1: Test operational requirements\\}

Initial state: N/A

Input: N/A

Expected Output: Game can be played on various operating systems

Result: Passed. However, the batch file only works on Windows. To run the code on MacOS, it is necessary to either compile source code, or to run the associated JAR file through a command.

\subsection{Maintainability and Support}

\paragraph{NFR-MS-1: Test maintainability requirements\\}

Initial state: N/A

Input: N/A

Expected Output: Well-documented source code, game rules and description on how to run the game for users

Results: Passed. All the source code for the game has Doxygen and JavaDoc comments. ReadME files have instructions on running the game.

\subsection{Cultural}

\paragraph{NFR-C-1: Test cultural requirements\\}

Initial state: N/A

Input: N/A

Expected Output: No offensive images or text in source code or displayed to users.

Results: Passed. Manually tested that there are no offensive images or text.

\paragraph{NFR-C-2: Test cultural requirements\\}

Initial state: N/A

Input: N/A

Expected Output: Game console output is all in English

Results: Passed. Checked that everything is written in English in the implementation, documentation, and anything visible to the user.

\subsection{Legal}

\paragraph{NFR-L-1: Test legal requirements\\}

Initial state: N/A

Input: N/A

Expected Output: Project adheres to copyright properties

Results: Passed. Checked the MIT License requirements and that the project adhered to all the requirements. 

\section{Comparison to Existing Implementation}	
These tests compare the program to the Go implementation.
\begin{itemize}
    \item NFR-LF-1 Test Game Appearance
    \item NFR-MS-1 Test Maintainability Requirements
\end{itemize}

\section{Unit Testing}
Unit tests were written with JUnit framework. 

\section{Changes Due to Testing}
Through the use of system testing by playing the game, errors in the code were discovered. These faults were fixed, and additional tests were created to ensure that the faults were fixed. 

An example of a change was preventing either participant in the game from drawing from the stock pile if the stock pile was empty. Tests FR-CP-8 and FR-GO-12 were created to assert that neither player draws from an empty stock pile.

Another example test that was added after system testing was FR-CP-7. An EmptyStackException was being thrown, as the algorithm for the computer attempted to peek on the top of the stack after the one card on it had been removed. This was patched and a new test was added to confirm the fix.

During system testing, it was apparent that testers were prone to making mistakes in typing user inputs, which would often lead to the program experiencing formatting exceptions and crashing. To address this issue and improve robustness, several different methods were implemented to address invalid user inputs, such as implementing guards, or taking advantage of exceptions raised by the program to prompt the user again for a valid input.

An IllegalArgumentException added to the constructor of the Card class to handle situations when an invalid rank was used to instantiate a Card object. 

Changes were made to the SortByRank class to ensure that melds will always be displayed in a certain order.

\section{Automated Testing}
Automated testing was used for this project using JUnit. Most of the tests performed at the unit level were automated tests as well as some integration tests which combined some components together. 

\section{Trace to Requirements}
\begin{table}[H]
    \centering
    \caption{Traceability Matrix: Functional Requirement}
    \begin{adjustbox}{max width=0.7\paperwidth}
    \begin{tabular}{l|ccccccccccc}
        \textbf{Test IDs} & \multicolumn{7}{c}{\textbf{Requirement IDs}}\\
        \hline
        ~ & \textbf{BE1} & \textbf{BE2} & \textbf{BE3} & \textbf{BE4} & \textbf{BE5} & \textbf{BE6} & \textbf{BE7} & \textbf{BE8}\\
        \textbf{FR-C-1}    & X & X & X & X & X & X & ~ & ~\\
        \textbf{FR-C-2}    & X & X & X & X & X & X & ~ & ~\\
        \textbf{FR-C-3}    & X & X & X & X & X & X & ~ & ~\\
        \textbf{FR-C-4}    & X & X & X & X & X & X & ~ & ~\\
        \textbf{FR-C-5}    & X & X & X & X & X & X & ~ & ~\\
        \textbf{FR-C-6}    & X & X & X & X & X & X & ~ & ~\\
        \textbf{FR-C-7}    & X & X & X & X & X & X & ~ & ~\\
        \textbf{FR-C-8}    & X & X & X & X & X & X & ~ & ~\\
        \textbf{FR-DP-1}   & ~ & ~ & X & ~ & ~ & X & ~ & ~\\
        \textbf{FR-DP-2}   & ~ & ~ & X & ~ & ~ & X & ~ & ~\\
        \textbf{FR-SP-1}   & ~ & X & ~ & ~ & ~ & ~ & ~ & ~\\
        \textbf{FR-CP-1}   & ~ & X & ~ & ~ & ~ & ~ & ~ & ~\\
        \textbf{FR-CP-2}   & ~ & X & ~ & ~ & ~ & ~ & ~ & ~\\
        \textbf{FR-CP-3}   & ~ & X & ~ & ~ & ~ & ~ & ~ & ~\\
        \textbf{FR-CP-4}   & ~ & X & ~ & ~ & ~ & ~ & ~ & ~\\
        \textbf{FR-CP-5}   & ~ & X & ~ & ~ & ~ & ~ & ~ & ~\\
        \textbf{FR-CP-6}   & ~ & X & ~ & ~ & ~ & ~ & ~ & ~\\
        \textbf{FR-CP-7}   & ~ & X & ~ & ~ & ~ & ~ & ~ & ~\\
        \textbf{FR-CP-8}   & ~ & X & ~ & ~ & ~ & ~ & ~ & ~\\
        \textbf{FR-H-1}    & ~ & ~ & ~ & ~ & ~ & X & ~ & ~\\
        \textbf{FR-H-2}    & ~ & ~ & ~ & ~ & ~ & X & ~ & ~\\
        \textbf{FR-H-3}    & ~ & ~ & ~ & ~ & ~ & X & ~ & ~\\
        \textbf{FR-H-4}    & ~ & ~ & ~ & ~ & ~ & X & ~ & ~\\
        \textbf{FR-H-5}    & X & X & X & ~ & X & X & ~ & ~\\
        \textbf{FR-H-6}    & ~ & ~ & ~ & ~ & ~ & X & ~ & ~\\
        \textbf{FR-H-7}    & ~ & ~ & ~ & ~ & ~ & X & ~ & ~\\
        \textbf{FR-H-8}    & X & ~ & ~ & ~ & ~ & X & ~ & ~\\
        \textbf{FR-H-9}    & ~ & ~ & ~ & ~ & ~ & X & ~ & ~\\
        \textbf{FR-P-1}    & X & ~ & ~ & ~ & ~ & ~ & ~ & ~\\
        \textbf{FR-P-2}    & ~ & X & X & ~ & ~ & ~ & ~ & ~\\
        \textbf{FR-P-3}    & ~ & X & X & X & X & X & X & ~\\
    \end{tabular}
    \end{adjustbox}
    \label{Traceability Matrix: Functional Requirement}
\end{table}

\begin{table}[H]
    \centering
    \begin{adjustbox}{max width=0.7\paperwidth}
    \begin{tabular}{l|ccccccccccc}
        \textbf{Test IDs} & \multicolumn{7}{c}{\textbf{Requirement IDs}}\\
        \hline
        ~ & \textbf{BE1} & \textbf{BE2} & \textbf{BE3} & \textbf{BE4} & \textbf{BE5} & \textbf{BE6} & \textbf{BE7} & \textbf{BE8}\\
        \textbf{FR-P-4}    & ~ & ~ & ~ & ~ & X & ~ & ~ & ~\\
        \textbf{FR-P-5}    & ~ & ~ & ~ & ~ & ~ & ~ & X & ~\\
        \textbf{FR-P-6}    & ~ & ~ & ~ & X & ~ & ~ & ~ & ~\\
        \textbf{FR-P-7}    & ~ & ~ & ~ & ~ & ~ & X & ~ & ~\\
        \textbf{FR-P-8}    & ~ & ~ & ~ & ~ & ~ & X & ~ & ~\\
        \textbf{FR-P-9}    & ~ & ~ & ~ & ~ & ~ & ~ & X & ~\\
        \textbf{FR-P-10}   & ~ & ~ & ~ & ~ & X & ~ & X & ~\\
        \textbf{FR-P-11}   & ~ & ~ & ~ & ~ & X & ~ & X & ~\\
        \textbf{FR-P-12}   & ~ & ~ & ~ & X & ~ & ~ & X & ~\\
        \textbf{FR-P-13}   & X & ~ & ~ & ~ & ~ & ~ & ~ & ~\\
        \textbf{FR-P-14}   & X & ~ & ~ & ~ & X & ~ & ~ & ~\\
        \textbf{FR-P-15}   & ~ & ~ & ~ & X & ~ & ~ & ~ & ~\\
        \textbf{FR-P-16}   & X & ~ & ~ & ~ & X & ~ & ~ & ~\\
        \textbf{FR-M-1}    & ~ & ~ & ~ & X & ~ & ~ & X & ~\\
        \textbf{FR-M-2}    & ~ & ~ & ~ & X & ~ & ~ & X & ~\\
        \textbf{FR-M-3}    & ~ & ~ & ~ & X & ~ & ~ & X & ~\\
        \textbf{FR-M-4}    & ~ & ~ & ~ & X & ~ & ~ & X & ~\\
        \textbf{FR-M-5}    & ~ & ~ & ~ & X & ~ & ~ & X & ~\\
        \textbf{FR-M-6}    & ~ & ~ & ~ & X & ~ & ~ & X & ~\\
        \textbf{FR-UIO-1}  & ~ & ~ & ~ & ~ & ~ & ~ & X & ~\\
        \textbf{FR-UIO-2}  & ~ & ~ & ~ & ~ & ~ & ~ & X & ~\\
        \textbf{FR-UIO-3}  & X & ~ & ~ & ~ & ~ & ~ & ~ & ~\\
        \textbf{FR-UIO-4}  & X & ~ & ~ & ~ & ~ & ~ & ~ & ~\\
        \textbf{FR-UIO-5}  & ~ & X & X & X & X & ~ & X & ~\\
        \textbf{FR-UIO-6}  & ~ & X & X & X & X & ~ & X & ~\\
        \textbf{FR-UIO-7}  & X & ~ & ~ & ~ & ~ & ~ & ~ & ~\\
        \textbf{FR-GO-1}   & ~ & ~ & ~ & ~ & ~ & ~ & X & ~\\
        \textbf{FR-GO-2}   & ~ & ~ & ~ & ~ & ~ & ~ & X & ~\\
        \textbf{FR-GO-3}   & ~ & ~ & ~ & ~ & ~ & ~ & X & ~\\
        \textbf{FR-GO-4}   & X & ~ & ~ & ~ & ~ & ~ & ~ & X\\
        \textbf{FR-GO-5}   & X & ~ & ~ & ~ & ~ & ~ & ~ & X\\
        \textbf{FR-GO-6}   & X & ~ & ~ & ~ & ~ & ~ & ~ & ~\\
        \textbf{FR-GO-7}   & X & ~ & ~ & ~ & ~ & ~ & ~ & ~\\
        \textbf{FR-GO-8}   & ~ & ~ & ~ & ~ & ~ & ~ & ~ & X\\
        \textbf{FR-GO-9}   & ~ & X & X & X & X & ~ & X & ~\\
        \textbf{FR-GO-10}  & X & ~ & ~ & ~ & ~ & ~ & ~ & ~\\
    \end{tabular}
    \end{adjustbox}
\end{table}

\begin{table}[H]
    \centering
    \begin{adjustbox}{max width=0.7\paperwidth}
    \begin{tabular}{l|ccccccccccc}
        \textbf{Test IDs} & \multicolumn{7}{c}{\textbf{Requirement IDs}}\\
        \hline
        ~ & \textbf{BE1} & \textbf{BE2} & \textbf{BE3} & \textbf{BE4} & \textbf{BE5} & \textbf{BE6} & \textbf{BE7} & \textbf{BE8}\\
        \textbf{FR-GO-11}  & X & ~ & ~ & ~ & ~ & ~ & ~ & ~\\
        \textbf{FR-GO-12}  & ~ & X & ~ & ~ & ~ & ~ & ~ & ~\\
    \end{tabular}
    \end{adjustbox}
\end{table}

\begin{table}[H]
    \centering
    \caption{Traceability Matrix: Non-Functional Requirement}
    \begin{adjustbox}{max width=0.7\paperwidth}
    \begin{tabular}{l|ccccccccccccccc}
        \textbf{Test IDs} & \multicolumn{9}{c}{\textbf{Requirement IDs}}\\
        \hline
        ~ & \textbf{LF1} & \textbf{UH1} & \textbf{P1} & \textbf{P2} & \textbf{OE1}
        & \textbf{MS1} & \textbf{C1} & \textbf{C2} & \textbf{L1}\\
        \textbf{NFR-LF-1}  & X & ~ & ~ & ~ & ~ & ~ & ~ & ~ & ~\\
        \textbf{NFR-UH-1}  & ~ & X & ~ & ~ & ~ & ~ & ~ & ~ & ~\\
        \textbf{NFR-P-1}   & ~ & ~ & X & ~ & ~ & ~ & ~ & ~ & ~\\
        \textbf{NFR-P-2}   & ~ & ~ & ~ & X & ~ & ~ & ~ & ~ & ~\\
        \textbf{NFR-OE-1}  & ~ & ~ & ~ & ~ & X & ~ & ~ & ~ & ~\\
        \textbf{NFR-MS-1}  & ~ & ~ & ~ & ~ & ~ & X & ~ & ~ & ~\\
        \textbf{NFR-C-1}   & ~ & ~ & ~ & ~ & ~ & ~ & X & ~ & ~\\
        \textbf{NFR-C-2}   & ~ & ~ & ~ & ~ & ~ & ~ & ~ & X & ~\\
        \textbf{NFR-L-1}   & ~ & ~ & ~ & ~ & ~ & ~ & ~ & ~ & X\\
    \end{tabular}
    \end{adjustbox}
    \label{Traceability Matrix: Non-Functional Requirement}
\end{table}

\newpage
\section{Trace to Modules}
The module IDs are from the Rummy For Dummies Module Guide.
\begin{table}[H]
    \centering
    \caption{Traceability Matrix: Modules}
    \begin{adjustbox}{max width=0.7\paperwidth}
    \begin{tabular}{l|ccccccccccc}
        \textbf{Test IDs} & \multicolumn{7}{c}{\textbf{Module IDs}}\\
        \hline
        ~ & \textbf{M1} & \textbf{M2} & \textbf{M3} & \textbf{M4} & \textbf{M5} & \textbf{M6} & \textbf{M7} & \textbf{M8} & \textbf{M9} & \textbf{M10}\\
        \textbf{FR-C-1}    & ~ & ~ & ~ & ~ & ~ & ~ & X & ~ & ~ & ~\\
        \textbf{FR-C-2}    & ~ & ~ & ~ & ~ & ~ & ~ & X & ~ & ~ & ~\\
        \textbf{FR-C-3}    & ~ & ~ & ~ & ~ & ~ & ~ & X & ~ & ~ & ~\\
        \textbf{FR-C-4}    & ~ & ~ & ~ & ~ & ~ & ~ & X & ~ & ~ & ~\\
        \textbf{FR-C-5}    & ~ & ~ & ~ & ~ & ~ & ~ & X & ~ & ~ & ~\\
        \textbf{FR-C-6}    & ~ & ~ & ~ & ~ & ~ & ~ & X & ~ & ~ & ~\\
        \textbf{FR-C-7}    & ~ & ~ & ~ & ~ & ~ & ~ & X & ~ & ~ & ~\\
        \textbf{FR-C-8}    & ~ & ~ & ~ & ~ & ~ & ~ & X & ~ & ~ & ~\\
        \textbf{FR-DP-1}   & ~ & ~ & ~ & ~ & ~ & X & ~ & ~ & ~ & ~\\
        \textbf{FR-DP-2}   & ~ & ~ & ~ & ~ & ~ & X & ~ & ~ & ~ & ~\\
        \textbf{FR-SP-1}   & ~ & ~ & ~ & ~ & X & ~ & ~ & ~ & ~ & ~\\
        \textbf{FR-CP-1}   & ~ & ~ & ~ & X & ~ & ~ & ~ & ~ & ~ & ~\\
        \textbf{FR-CP-2}   & ~ & ~ & ~ & X & ~ & ~ & ~ & ~ & ~ & ~\\
        \textbf{FR-CP-3}   & ~ & ~ & ~ & X & ~ & ~ & ~ & ~ & ~ & ~\\
        \textbf{FR-CP-4}   & ~ & ~ & ~ & X & ~ & ~ & ~ & ~ & ~ & ~\\
        \textbf{FR-CP-5}   & ~ & ~ & ~ & X & ~ & ~ & ~ & ~ & ~ & ~\\
        \textbf{FR-CP-6}   & ~ & ~ & ~ & X & ~ & ~ & ~ & ~ & ~ & ~\\
        \textbf{FR-CP-7}   & ~ & ~ & ~ & X & ~ & ~ & ~ & ~ & ~ & ~\\
        \textbf{FR-CP-8}   & ~ & ~ & ~ & X & ~ & ~ & ~ & ~ & ~ & ~\\
        \textbf{FR-H-1}    & ~ & ~ & ~ & ~ & ~ & ~ & ~ & X & ~ & ~\\
        \textbf{FR-H-2}    & ~ & ~ & ~ & ~ & ~ & ~ & ~ & X & ~ & ~\\
        \textbf{FR-H-3}    & ~ & ~ & ~ & ~ & ~ & ~ & ~ & X & ~ & ~\\
        \textbf{FR-H-4}    & ~ & ~ & ~ & ~ & ~ & ~ & ~ & X & ~ & ~\\
        \textbf{FR-H-5}    & ~ & ~ & ~ & ~ & ~ & ~ & ~ & X & ~ & ~\\
        \textbf{FR-H-6}    & ~ & ~ & ~ & ~ & ~ & ~ & ~ & X & ~ & ~\\
        \textbf{FR-H-7}    & ~ & ~ & ~ & ~ & ~ & ~ & ~ & X & ~ & ~\\
        \textbf{FR-H-8}    & ~ & ~ & ~ & ~ & ~ & ~ & ~ & X & ~ & ~\\
        \textbf{FR-H-9}    & ~ & ~ & ~ & ~ & ~ & ~ & ~ & X & ~ & ~\\
        \textbf{FR-P-1}    & ~ & ~ & ~ & ~ & ~ & ~ & ~ & ~ & ~ & X\\
        \textbf{FR-P-2}    & ~ & ~ & ~ & ~ & ~ & ~ & ~ & ~ & ~ & X\\
    \end{tabular}
    \end{adjustbox}
    \label{Traceability Matrix: Functional Requirement}
\end{table}

\begin{table}[H]
    \centering
    \begin{adjustbox}{max width=0.7\paperwidth}
    \begin{tabular}{l|ccccccccccc}
        \textbf{Test IDs} & \multicolumn{7}{c}{\textbf{Module IDs}}\\
        \hline
        ~ & \textbf{M1} & \textbf{M2} & \textbf{M3} & \textbf{M4} & \textbf{M5} & \textbf{M6} & \textbf{M7} & \textbf{M8} & \textbf{M9} & \textbf{M10}\\
        \textbf{FR-P-3}    & ~ & ~ & ~ & ~ & ~ & ~ & ~ & ~ & ~ & X\\
        \textbf{FR-P-4}    & ~ & ~ & ~ & ~ & ~ & ~ & ~ & ~ & ~ & X\\
        \textbf{FR-P-5}    & ~ & ~ & ~ & ~ & ~ & ~ & ~ & ~ & ~ & X\\
        \textbf{FR-P-6}    & ~ & ~ & ~ & ~ & ~ & ~ & ~ & ~ & ~ & X\\
        \textbf{FR-P-7}    & ~ & ~ & ~ & ~ & ~ & ~ & ~ & ~ & ~ & X\\
        \textbf{FR-P-8}    & ~ & ~ & ~ & ~ & ~ & ~ & ~ & ~ & ~ & X\\
        \textbf{FR-P-9}    & ~ & ~ & ~ & ~ & ~ & ~ & ~ & ~ & ~ & X\\
        \textbf{FR-P-10}    & ~ & ~ & ~ & ~ & ~ & ~ & ~ & ~ & ~ & X\\
        \textbf{FR-P-11}    & ~ & ~ & ~ & ~ & ~ & ~ & ~ & ~ & ~ & X\\
        \textbf{FR-P-12}    & ~ & ~ & ~ & ~ & ~ & ~ & ~ & ~ & ~ & X\\
        \textbf{FR-P-13}    & ~ & ~ & ~ & ~ & ~ & ~ & ~ & ~ & ~ & X\\
        \textbf{FR-P-14}    & ~ & ~ & ~ & ~ & ~ & ~ & ~ & ~ & ~ & X\\
        \textbf{FR-P-15}    & ~ & ~ & ~ & ~ & ~ & ~ & ~ & ~ & ~ & X\\
        \textbf{FR-P-16}    & ~ & ~ & ~ & ~ & ~ & ~ & ~ & ~ & ~ & X\\
        \textbf{FR-M-1}    & ~ & ~ & ~ & ~ & ~ & ~ & ~ & ~ & X & ~\\
        \textbf{FR-M-2}    & ~ & ~ & ~ & ~ & ~ & ~ & ~ & ~ & X & ~\\
        \textbf{FR-M-3}    & ~ & ~ & ~ & ~ & ~ & ~ & ~ & ~ & X & ~\\
        \textbf{FR-M-4}    & ~ & ~ & ~ & ~ & ~ & ~ & ~ & ~ & X & ~\\
        \textbf{FR-M-5}    & ~ & ~ & ~ & ~ & ~ & ~ & ~ & ~ & X & ~\\
        \textbf{FR-M-6}    & ~ & ~ & ~ & ~ & ~ & ~ & ~ & ~ & X & ~\\
        \textbf{FR-UIO-1}  & ~ & ~ & X & ~ & ~ & ~ & ~ & ~ & ~ & ~\\
        \textbf{FR-UIO-2}  & ~ & ~ & X & ~ & ~ & ~ & ~ & ~ & ~ & ~\\
        \textbf{FR-UIO-3}  & ~ & ~ & X & ~ & ~ & ~ & ~ & ~ & ~ & ~\\
        \textbf{FR-UIO-4}  & ~ & ~ & X & ~ & ~ & ~ & ~ & ~ & ~ & ~\\
        \textbf{FR-UIO-5}  & ~ & ~ & X & ~ & ~ & ~ & ~ & ~ & ~ & ~\\
        \textbf{FR-UIO-6}  & ~ & ~ & X & ~ & ~ & ~ & ~ & ~ & ~ & ~\\
        \textbf{FR-UIO-7}  & ~ & ~ & X & ~ & ~ & ~ & ~ & ~ & ~ & ~\\
        \textbf{FR-GO-1}   & ~ & X & ~ & ~ & ~ & ~ & ~ & ~ & ~ & ~\\
        \textbf{FR-GO-2}   & ~ & X & ~ & ~ & ~ & ~ & ~ & ~ & ~ & ~\\
        \textbf{FR-GO-3}   & ~ & X & ~ & ~ & ~ & ~ & ~ & ~ & ~ & ~\\
        \textbf{FR-GO-4}   & ~ & X & ~ & ~ & ~ & ~ & ~ & ~ & ~ & ~\\
        \textbf{FR-GO-5}   & ~ & X & ~ & ~ & ~ & ~ & ~ & ~ & ~ & ~\\
        \textbf{FR-GO-6}   & ~ & X & ~ & ~ & ~ & ~ & ~ & ~ & ~ & ~\\
    \end{tabular}
    \end{adjustbox}
\end{table}

\begin{table}[H]
    \centering
    \begin{adjustbox}{max width=0.7\paperwidth}
    \begin{tabular}{l|ccccccccccccccc}
        \textbf{Test IDs} & \multicolumn{7}{c}{\textbf{Module IDs}}\\
        \hline
        ~ & \textbf{M1} & \textbf{M2} & \textbf{M3} & \textbf{M4} & \textbf{M5} & \textbf{M6} & \textbf{M7} & \textbf{M8} & \textbf{M9} & \textbf{M10}\\
        \textbf{FR-GO-7}   & ~ & X & ~ & ~ & ~ & ~ & ~ & ~ & ~ & ~\\
        \textbf{FR-GO-8}   & ~ & X & ~ & ~ & ~ & ~ & ~ & ~ & ~ & ~\\
        \textbf{FR-GO-9}   & ~ & X & ~ & ~ & ~ & ~ & ~ & ~ & ~ & ~\\
        \textbf{FR-GO-10}  & ~ & X & ~ & ~ & ~ & ~ & ~ & ~ & ~ & ~\\
        \textbf{FR-GO-11}  & ~ & X & ~ & ~ & ~ & ~ & ~ & ~ & ~ & ~\\
        \textbf{FR-GO-12}  & ~ & X & ~ & ~ & ~ & ~ & ~ & ~ & ~ & ~\\
        \textbf{NFR-LF-1}  & ~ & X & ~ & ~ & ~ & X & ~ & X & ~ & ~ \\
        \textbf{NFR-UH-1}  & ~ & X & X & ~ & ~ & X & ~ & X & ~ & ~\\
        \textbf{NFR-P-1}   & ~ & X & ~ & X & ~ & ~ & ~ & ~ & X & ~\\
        \textbf{NFR-P-2}   & ~ & X & X & ~ & ~ & ~ & ~ & ~ & ~ & ~\\
        \textbf{NFR-OE-1}  & X & ~ & ~ & ~ & ~ & ~ & ~ & ~ & ~ & ~\\
        \textbf{NFR-MS-1}  & ~ & X & X & X & X & X & X & X & X & X\\
        \textbf{NFR-C-1}   & ~ & X & ~ & ~ & ~ & X & ~ & X & ~ & ~ \\
        \textbf{NFR-C-2}   & ~ & X & ~ & ~ & ~ & X & ~ & X & ~ & ~ \\
        \textbf{NFR-L-1}   & ~ & X & X & X & X & X & X & X & X & X\\
    \end{tabular}
    \end{adjustbox}
    \label{Traceability Matrix: Non-Functional Requirement}
\end{table}

\section{Code Coverage Metrics}
The tests have produced a 90\% code coverage. This is based on the overlapping coverage of the modules during testing. To see this overlapping coverage, refer to the Trace to Modules table.

\bibliographystyle{plainnat}

\bibliography{SRS}

\end{document}