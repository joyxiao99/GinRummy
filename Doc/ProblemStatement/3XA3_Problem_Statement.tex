\documentclass[12pt]{article}
\usepackage{indentfirst}

\title{SFWRENG 3XA3 Problem Statement}
\author{
	Lab 02 Group 7, Rummy for Dummies
		\\ Joy Xiao, xiaoz18
        \\ Benson Hall, hallb8
        \\ Smita Singh, sings59
}
\date{January 27, 2021}

\begin{document}

\maketitle

\section{Problem Statement}
\subsection*{What problem is being solved?}
With the ongoing pandemic, people are seeking new ways to entertain themselves from the comfort and safety of their home. The intent is to recreate a single-player card game that is normally played with several people. This is to give users the experience of playing a multiplayer card game within their own homes.

\subsection*{Why is it an important problem?}
Rummy is a popular card game. Although the currently-available Rummy apps on app stores are free, they have many in-game advertisements. Many users may find this an annoyance. The software being developed is to be played on the local machine, to avoid advertisements and the need for internet access.

The software will be an improved version of Rummy. The game will require little memory to play. The project will also be more user-friendly compared to the open-source implementation, as it will be extended to be accessible to the general public, even if they do not have a technical background.

\subsection*{What is the context of the problem being solved?}
The software will run on a desktop environment. A majority of desktop computers come with a command line and a Java Runtime Environment (JRE). Users can play the game through the command line interface. This creates a more accessible implementation that allows users to play the game offline.

A group of stakeholders would be users that are familiar with Rummy. Allowing the game to be played through the desktop will give more people access to enjoying the game. In particular, card game enthusiasts would have a great deal of interest in this software because they actively seek out various card games to play. Another stakeholder are the developers. They are responsible for producing, testing and maintaining the project. For this course, Dr. Bokhari and the TAs are the main stakeholders of this project.

\end{document}